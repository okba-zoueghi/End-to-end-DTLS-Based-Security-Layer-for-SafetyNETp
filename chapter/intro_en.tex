\chapter*{General Introduction}
\addcontentsline{toc}{chapter}{General Introduction}
\markboth{General Introduction}{}

Nowadays, the industry is going more and more towards automation technology which is about automating industrial
plants and their machines to such an extent that they independently do a job without the involvement of people.
Although humans can control machines and industrial plants, they would not intervene in their work process.
Thus, when a work process starts, human errors are eliminated. As a result, there is a significant
increase in the quality of the products created.

The modern automated systems and field instruments currently being used are no longer mere mechanical devices,
they are mostly controlled and managed by digital computers\cite{koscher2010experimental}. The majority of
those systems and devices have not been designed with security measures\cite{zurawski2014industrial}. In fact,
many years ago, the motivation and the interest for hackers to compromise them was little or even inexistent.
However, nowadays with new concepts such as Industrial Ethernet, Connected-Cars, Autonomous Vehicles and \ac{IoT}, the motivation for attacking those systems is increasing significantly.

Recently, multiple studies \cite{koscher2010experimental} demonstrated that the risk exists for automated systems to be
infiltrated by malicious attackers that are potentially able to gain access via several methods. Despite this concrete threat,
most of the industrial communication protocols do not have any built-in provisions to prevent or mitigate these attacks and
no security aspect is part of the software or hardware architecture development
process.

Current industrial communication protocols, including SafetyNETp, SafetyBUSp, \ac{CAN}, and
\ac{LIN}, are vulnerable to attacks. Actually, they enable unauthorized access
in a relatively straightforward manner given that all the communications are performed with no authentication \cite{zurawski2014industrial}.
However, we are convinced that security can be taken into account in the early phases of
the development cycle of automated systems, both by implementing security protocols
which can assure the authentication of the sender, the integrity of the message and the availability
of the network, as well as by enforcing software programming standards that prevent software defects.

\ac{TLS} is the most widely deployed protocol for securing network traffic. It is widely used for
protecting Web traffic, for instance, \ac{HTTPS} the secure version of \ac{HTTP} is secured using \ac{TLS}. \ac{TLS} is also
used for securing e-mail protocols such as \ac{IMAP} and \ac{POP}. The primary advantage of \ac{TLS} is
that it provides in a transparent way confidentiality, integrity, and authenticity to a given connection-oriented channel. Thus, it is
easy to secure an application protocol by inserting \ac{TLS} between the application layer and the transport
layer \cite{rfc5246}. However, \ac{TLS} must run over a reliable transport channel typically the \ac{TCP}. Therefore, it cannot be
used to secure unreliable datagram traffic. The requirement for datagram semantics automatically prohibits the
use of \ac{TLS} and for this purpose, there exists \ac{DTLS} which provides \ac{TLS}'s services for unreliable datagram
protocols such as the \ac{UDP} \cite{rfc6347}.

Alongside \ac{TLS}, \ac{IPsec} is also a commonly used protocol for securing network traffic. It is implemented
in the network layer and it provides security for both IPv4 and IPv6 traffics. Operating on the network layer
makes \ac{IPsec} transparent and independent from the application layer, this, in fact, eases the security
management in multiapplication environments. \ac{IPsec} is mostly used for creating a \ac{VPN}.

In this context, part of the funding program from \ac{DAAD}, we have integrated the \ac{ivESK} to perform our end of studies project.
Our project consists of designing and developing a security layer for SafetyNETp which is a protocol for Ethernet-based Fieldbus communication
in automation technology. Moreover, after the development, a test of the implementation is required to get an idea about
the potential behavior of secured SafetyNETp networks in a realistic environment.

During the first chapter, we see a brief presentation of the host organism then we present the
problem description as well as the objective and the proposed solution. Throughout the second chapter, we present an overview of industrial networks and the target protocol of our project (SafetyNETp),
thereafter we move to present the available security protocols for securing network traffic and finally discuss our choice.
In the third chapter we go through a brief security analysis and we outline the requirements specification and analysis.
The fourth chapter is dedicated to the design of our security layer, it structured into two parts, the first part
discusses the global design and the second part discusses the detailed design.
Along the fifth and the last chapter, we present some implementation details and we test the performance of
our solution.

% In this context, part of the funding program from \ac{DAAD}, we have integrated the \ac{ivESK}, which was formed to focus on wired and wireless networks of
% embedded systems, as well as their interconnection in "Cyber-Physical Systems" (CPS) and data
% security and privacy aspects. \ac{ivESK} is especially active in the area of design and implementation of integrated
% security architectures for communications solutions using embedded systems \cite{ivESK} where our end of studies project is taking place.
% In fact, this project consists of developing a security layer for SafetyNETp which is a protocol for Ethernet-based Fieldbus communication in automation technology.
% Moreover, at the end of this project, a test of the implementation is required to get an idea about the potential behavior of secured SafetyNETp networks in a realistic environment.
