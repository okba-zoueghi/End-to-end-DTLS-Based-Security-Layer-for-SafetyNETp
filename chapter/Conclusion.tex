%_____________________________________________________________________________________
%
%       Filename:  Conclusion.tex
%
%    Description:  Thesis Template HS Offenburg
%
%
%         Author:  Okba ZOUEGHI, okba.zoueghi@gmail.com
%     Supervisor:  Andreas Walz, Chadlia Jerad
%   Organization:  HS Offenburg, Offenburg, Germany
%
%_____________________________________________________________________________________

\chapter*{General Conclusion and Perspectives}
\addcontentsline{toc}{chapter}{General Conclusion}
\markboth{General Conclusion and Perspectives}{}

Today, most of the currently available industrial networks focus mainly on meeting real-time requirements and are designed and developed
without any built-in security. SafetyNETp is one of the well known and most used Industrial Ethernet protocols, although it does not
provide any special security measures. SafetyNETp based networks are vulnerable to attacks as it enables unauthorized access in a
relatively straightforward manner given that all the communications are performed with no authentication, no integrity checks, and no encryption.

Designing and implementing a security layer for SafetyNETp from scratch is a difficult and very error-prone task.
That's why we were interested in our project by the available protocols which are already tested and assessed by the internet community.
DTLS and IPsec were the candidate protocols to be used. In fact, our protocol of choice was DTLS, this is mainly because DTLS is situated at the session layer
whereas IPsec is situated at the network layer. Actually, this makes DTLS a lot easier to integrate.

The implemented DTLS based security layer provides confidentiality, integrity, and authenticity respectively by
using encryption, MAC algorithms, and certificates. It is obvious that a time and a data overhead will be introduced by
running those security mechanisms. In fact, the performance tests have shown that the time needed for sending
and receiving a message over the security layer is around 7 milliseconds using
TLS\_RSA\_WITH\_AES\_128\_CBC\_SHA256 cipher suite.

SafetyNETp provides two communication models compatible with each other which are RTFN and RTFL. RTFN is
used when a cycle time over 1 millisecond is sufficient whereas RTFL provides a cycle time at microsecond
level. The current design and implementation of the security layer concerns only RTFN and does not provide
any security to RTFL. Therefore, the communication between the secured RTFN and RTFL is not
possible. In fact, Securing RTFL needs entirely a new design, moreover, as RTFL time constraints are more
tight, integrating a security layer may affect significantly its cycle time.
