%-------------------------------------------------------------------------------
%
%       Filename:  preamble.tex
%
%    Description:  Complete Latex preamble is here.
%                     
%
%        Version:  0.1
%        Created:  29.04.2016
%       Revision:  none
%
%         Author:  M.Sc. Oliver Kehret, oliver.kehret@hs-offenburg.de
%   Organization:  ivESK, Offenburg University, Germany
%      Copyright:  Copyright (c) 2016, M.Sc. Oliver Kehret
%
%-------------------------------------------------------------------------------
%
\newcommand{\mylanguage}{\secondlanguage,\firstlanguage}
\documentclass[%
    fontsize=12pt,
	paper=A4,
	parskip=half,
	DIV=24,%calc
	headinclude=true,		
	footinclude=true,
	open=right,
	appendixprefix=true,% include appendix?
	bibliography=totoc,% include an unnumbered unit of bibliography to the table
                       % of contents 
    draft=\mydraft,
	BCOR=00mm,% binding correction (depends on how you bind
	             % the resulting printout.
	\mylaterality,% alternative: twoside
    \mylanguage,
    plainheadsepline=true,
    plainfootsepline=true, 
]{scrbook}
%-------------------------------------------------------------------------------
% ifthen 
%-------------------------------------------------------------------------------
\usepackage{ifthen}                                         
% pre-define ifthen-boolean variables:
\newboolean{report}
\newboolean{myhyperref}
\newboolean{myaddlistoftodos}
\newboolean{withgnuplot}
\setboolean{withgnuplot}{false}
%-------------------------------------------------------------------------------
%
%
%-------------------------------------------------------------------------------
% General presentation
%
%-------------------------------------------------------------------------------
% full utf8 character set
\usepackage[utf8]{inputenc}
% language adaptions, change in main.tex (english, ngerman, american)
\usepackage[\mylanguage]{babel}									
\usepackage{iflang}
%
\usepackage{lmodern}
% encode characters better, so that they can be copied out of .pdf in one piece
\usepackage[T1]{fontenc}    
% nicer quotes  
%
\usepackage[onehalfspacing]{setspace}
%
\IfLanguageName{ngerman}{%
\usepackage[%
            autostyle,          % adapts language setting
            german=quotes,
            strict,             % turns warnings into errors 
]{csquotes}
}{}
\IfLanguageName{american}{%
\usepackage[%
            autostyle,          % adapts language setting
            strict,             % turns warnings into errors 
            english=american    % use american quotes style
]{csquotes}
}{}
%
%-------------------------------------------------------------------------------
% Color definitions
%-------------------------------------------------------------------------------
\usepackage[table,usenames,dvipsnames]{xcolor}		
\definecolor{light-gray}{gray}{0.95}	% define some colors
\definecolor{dark-green}{rgb}{0,0.6,0}
\definecolor{iveskblue}{RGB}{0,63,97}
%
%-------------------------------------------------------------------------------
% ivESK style
%
%-------------------------------------------------------------------------------
\usepackage{scrlayer-scrpage}
\usepackage{scrhack}%silence warning old packages
\pagestyle{plain.scrheadings}% or scrheadings
\clearscrheadfoot
\setkomafont{pageheadfoot}{%
  \scriptsize% 
}
 \usepackage{fancyhdr} 
\fancyhf{}
%supprimelesentetesetpiedsexistant
%\fancyhead[LE,LO]{\bfseries{Projet de fin d'?tudes}}
\fancyhead[RE,RO]{\bfseries\leftmark}
\fancyfoot[LE,LO]{\textit {Institut f{\"u}r verl{\"a}ssliche Embedded Systems und Kommunikationselektronik (ivESK)\\National School of Computer Science (ENSI)}} \fancyfoot[RE,RO]{\bfseries\thepage}
\renewcommand{\headrulewidth}{0.5pt}
\renewcommand{\footrulewidth}{0.5pt}
\addtolength{\headheight}{0.5pt}
%\addtolength{\footheight}{0.5pt} %espacepourlefilet
\fancypagestyle{plain}{%pages de tetes de chapitre
\fancyhead{}%supprimelentete
\renewcommand{\headrulewidth}{0pt}%etlefilet
} \pagestyle{fancy}
\usepackage{lastpage}
\rofoot*{Page \thepage~of~\pageref{LastPage}}
%\KOMAoptions{headheight=0cm}
%\KOMAoptions{footheight=0cm}
\KOMAoptions{headsepline=1pt:\textwidth}
\KOMAoptions{footsepline=1pt:\textwidth}
\setkomafont{headsepline}{\color{iveskblue}}
\setkomafont{footsepline}{\color{iveskblue}}% change colors of seperation lines
%
\RedeclareSectionCommand[
  beforeskip=1.5\baselineskip,
  afterskip=0.5\baselineskip]{chapter}
\RedeclareSectionCommand[
  beforeskip=1\baselineskip,
  afterskip=0.5\baselineskip]{section}
\RedeclareSectionCommand[
  beforeskip=0.1\baselineskip,
  afterskip=1pt]{subsection}
\RedeclareSectionCommand[
  beforeskip=2pt,
  afterskip=1pt]{subsubsection}
%
%-------------------------------------------------------------------------------
% includes pictures --> possible to include them in header/footer
%-------------------------------------------------------------------------------
%
\usepackage[pdftex]{graphicx}										
%-------------------------------------------------------------------------------
% footnotes
%-------------------------------------------------------------------------------
\usepackage[perpage, hang]{footmisc}% footnote options
%perpage: Nummerierung der Fußnoten auf jeder Seite neu beginnen
%hang: Fußnote über mehrere zeilen richtig einrücken
%
%-------------------------------------------------------------------------------
% placeins \FloatBarrier forces places of floats
%-------------------------------------------------------------------------------
\usepackage[verbose, %
section,% force for section automatically
]
{placeins}										
%
%-------------------------------------------------------------------------------
% bibliography with biber/biblatex
%
%-------------------------------------------------------------------------------
\usepackage[backend=biber, % using "biber" to compile references (instead of
                           % "biblatex") 
style=numeric-comp,% see biblatex documentation
style=numeric,% see biblatex documentation
backref=true,% create backlings from references to citations
natbib=true,% offering natbib-compatible commands
hyperref=true,%
sorting=none,% using hyperref-package references
]{biblatex}% remove, if using BibTeX instead of biblatex
\addbibresource{\mybiblatexfile} % remove, if using BibTeX instead of biblatex
%
%-------------------------------------------------------------------------------
% SIUNITSX -- simplified usage of SI-units
%-------------------------------------------------------------------------------
%
\usepackage[% 
            exponent-product=\cdot,% use \cdot instead * for exponent product
            binary-units=true,%
            load-configurations=binary,%
            load-configurations=abbreviations,%
]{siunitx}	              
%
%
%-------------------------------------------------------------------------------
% todonotes puts to-do-notes in the printed document if you want 
%-------------------------------------------------------------------------------
%
\IfLanguageName{american}{\usepackage[\mytodonotesoptions,english]{todonotes}}{}
\IfLanguageName{ngerman}{\usepackage[\mytodonotesoptions,ngerman]{todonotes}}{}
%
%-------------------------------------------------------------------------------
% Sourcecode printing
%-------------------------------------------------------------------------------
%
\usepackage{listings}				% include source code
									% ftp://ftp.tex.ac.uk/tex-archive/macros/latex/contrib/listings/listings.pdf
\lstset{% 							% options for representation of source code
backgroundcolor=\color{light-gray},% choose the background color; you must add \usepackage{color} or \usepackage{xcolor}
basicstyle=\footnotesize\ttfamily,        % the size of the fonts that are used for the code
breakatwhitespace=false,         % sets if automatic breaks should only happen at whitespace
breaklines=true,                 % sets automatic line breaking
captionpos=b,                    % sets the caption-position to bottom
commentstyle=\color{dark-green}, % comment style
deletekeywords={...},            % if you want to delete keywords from the given language
escapeinside={\%*}{*)},          % if you want to add LaTeX within your code
extendedchars=true,              % lets you use non-ASCII characters; for 8-bits encodings only, does not work with UTF-8
frame=lines,                     % adds a frame around the code
keepspaces=true,                 % keeps spaces in text, useful for keeping indentation of code (possibly needs columns=flexible)
keywordstyle=\color{blue},       % keyword style
language=C,                      % the language of the code
morekeywords={*,...},            % if you want to add more keywords to the set
numbers=left,                    % where to put the line-numbers; possible values are (none, left, right)
numbersep=8pt,                   % how far the line-numbers are from the code
numberstyle=\tiny\color{gray},   % the style that is used for the line-numbers
rulecolor=\color{black},         % if not set, the frame-color may be changed on line-breaks within not-black text (e.g. comments (green here))
showspaces=false,                % show spaces everywhere adding particular underscores; it overrides 'showstringspaces'
showstringspaces=false,          % underline spaces within strings only
showtabs=false,                  % show tabs within strings adding particular underscores
stepnumber=1,                    % the step between two line-numbers. If it's 1, each line will be numbered
stringstyle=\color{blue},        % string literal style
tabsize=2,                       % sets default tabsize to 2 spaces
title=\lstname,                  % show the filename of files included with \lstinputlisting; also try caption instead of title
%numberbychapter=false
}%
%
%-------------------------------------------------------------------------------
% Math 
%-------------------------------------------------------------------------------
%
\usepackage{amssymb,amstext,amsmath} % predefiened symbols e.g. \nparallel
\usepackage[%
            fleqn,%equations aligned in a fixed distance from the left
            tbtags, %where the equation number is placed here bottom or top
]{mathtools} % loads amsmath package
%
%-------------------------------------------------------------------------------
% Tables, figures etc.
%-------------------------------------------------------------------------------
%
\usepackage{hhline}
%
\usepackage{colortbl}
% nice rule's for tables try \toprule \midrule \bottomrule  
\usepackage{booktabs}
% set width of table and more
\usepackage{tabularx}										% creates tables
%
% rotate tables and figures
\usepackage{rotating}
%
% rotate tables and figures
\usepackage{multirow}
%
% define caption style
\addtokomafont{caption}{\small} 	% small captions
\usepackage[font=small,% 
width=0.9\textwidth,% 
format=plain,% 
labelfont=bf,%
]{caption}
\captionsetup[figure]{%
name=Fig.,%
labelfont=bf,%
textfont=it,%
}
\captionsetup[table]{%
name=Tab.,%
labelfont=bf,%
textfont=it,%
}
%
%
\usepackage{subfigure}
%-------------------------------------------------------------------------------
% list of acronym's
%-------------------------------------------------------------------------------
\usepackage[printonlyused]{acronym}
%
%-------------------------------------------------------------------------------
% some utility stuff
%
%-------------------------------------------------------------------------------
%
%
% improved typographical settings
\usepackage[%
    protrusion=true, %
    factor=900       %
]{microtype}
%
% switch of extra space after punctuation
\frenchspacing 
%
% switches to Palatino with small caps and old style figures
\usepackage[%
sc,%
osf,%
]{mathpazo}
%
% customize item look
\usepackage{enumitem}
% kills space between items
\setlist{noitemsep}%
\setlist[itemize,2]{label={$\circ$}} 
\setlist[itemize,3]{label={\tiny$\blacksquare$}} 

% For additional special characters available by \verb#\ding{}#
\usepackage{pifont}  % Sonderzeichen fuer Titelseite \ding{}
%
% This package is required for intelligent spacing after commands
\usepackage{xspace}
%
%
% This package offers strikethrough command \verb+\sout{foobar}+.
\usepackage[normalem]{ulem}
%
%
% Create framed, shaded, or differently highlighted regions that can 
% break across pages.  The environments defined are 
% \begin{itemize}
%   \item framed: ordinary frame box (\verb+\fbox+) with edge at margin
%   \item shaded: shaded background (\verb+\colorbox+) bleeding into margin
%   \item snugshade: similar
%   \item leftbar: thick vertical line in left margin
% \end{itemize}
\usepackage{framed}
%
% For example on title pages you might want to have a logo on the upper right
% corner of the first page (only). The package \texttt{eso-pic} is able to place
% things on absolute and relative positions on the whole page.
\usepackage{eso-pic} %
%
%-------------------------------------------------------------------------------
% Rename directories with babel package
%-------------------------------------------------------------------------------
\addto\captionsamerican{% Replace "american" with the language you use
  \renewcommand{\contentsname}%
    {Table of Contents}%
}
%
%-------------------------------------------------------------------------------
% Own Colors for header, captions etc.
%
%-------------------------------------------------------------------------------
\usepackage{tikz}
\usetikzlibrary{positioning,shapes,arrows}%
\tikzstyle{memblock} = [draw, fill=blue!20, rectangle, 
    minimum height=6em, minimum width=3em]%
\definecolor{mygray}{cmyk}{0,0,0,0.4}%
\definecolor{mydarkgray}{cmyk}{0,0,0,0.7}%
\definecolor{mylightgray}{cmyk}{0,0,0,0.1}%
%
%-------------------------------------------------------------------------------
% interface with gnuplot latex terminal
%-------------------------------------------------------------------------------
\ifthenelse{\boolean{withgnuplot}}%if
{
\usepackage{gnuplot-lua-tikz}
}{%
}%
%-------------------------------------------------------------------------------
%tikz flow chart
\tikzstyle{decision} = [diamond, draw, fill=blue!20, 
    text width=4.25em, text badly centered, node distance=2cm, inner sep=0pt]
\tikzstyle{block} = [rectangle, draw, fill=blue!20, 
    text width=2.5em, text centered, rounded corners, minimum height=1em]
\tikzstyle{line} = [draw, -latex']
\tikzstyle{cloud} = [draw, ellipse,fill=red!20, node distance=2cm,
    minimum height=1em]
%
%Plottiing
%-------------------------------------------------------------------------------
\usepackage{lipsum}%insert blindtext for testing
%-------------------------------------------------------------------------------
\usepackage[\firstlanguage]{isodate}% several date options
%-------------------------------------------------------------------------------
%
%-------------------------------------------------------------------------------
\usepackage{fancyhdr}
%\usepackage{supertabular}
\usepackage{array}
\usepackage{multirow}
\usepackage{longtable}
%\usepackage{algorithm}
%\usepackage{algorithmic}
\usepackage{float}
\usepackage{pgfplots}
\pgfplotsset{compat = newest}
\pgfkeys{/pgf/number format/use comma,/pgf/number format/set thousands separator
  = } 
%
% prevent club & widow penalty
\clubpenalty10000
\widowpenalty10000
\displaywidowpenalty10000
%
%
%-------------------------------------------------------------------------------
% pdfcompresslevel from 0 to 10; std is fine 
%-------------------------------------------------------------------------------
%
\pdfcompresslevel=6 
\usepackage{nameref}
\usepackage{epigraph}
\newcommand{\mychapter}[2]{
    \setcounter{chapter}{#1}
    \setcounter{section}{0}
    \chapter*{#2}
    \addcontentsline{toc}{chapter}{#2}
}
\usepackage{rotating}
\usepackage{arabtex}
\usepackage{utf8}