%-------------------------------------------------------------------------------
%
%       Filename:  history.tex
%
%    Description:  The authors and the report history are listed here.
%
%        Version:  0.1
%        Created:  29.04.2016
%       Revision:  none
%
%         Author:  M.Sc. Oliver Kehret, oliver.kehret@hs-offenburg.de
%   Organization:  ivESK, Offenburg University, Germany
%      Copyright:  Copyright (c) 2016, M.Sc. Oliver Kehret
%
%-------------------------------------------------------------------------------
%
% \chapter*{\myauthorchapter}
% \label{cha:authors}
% \addxcontentsline{toc}{chapter}{\myauthorchapter}
% %-------------------------------------------------------------------------------
% \begin{tabular}{l}
%   \myauthor    \hspace{5mm}       [youssef.rekik@ensi-uma.tn]
% \end{tabular}
% %
% %
% %
% %-------------------------------------------------------------------------------
% %
% \ifthenelse{\equal{\mylaterality}{oneside}}%if
% {{\let\clearpage\relax \chapter*{\myhistory}}}%then
% {{\let\cleardoublepage\relax \chapter*{\myhistory}}}%
% \addxcontentsline{toc}{chapter}{\myhistory}
% \label{cha:history}
% \begin{tabular}{|l|l|l|l|}
%   \hline
%   \rowcolor{iveskblue}
%   \mytableheadfont{Version} & \mytableheadfont{\myauthorchapter} & \mytableheadfont{\mydate} &\mytableheadfont{\mysummaryofchanges} \\
%   %\textbf{\textcolor{white}{Version}}   & \textbf{\myauthorchapter} & \textbf{\mydate} & \textbf{\mysummaryofchanges} \\
%   \hhline{|=|=|=|=|}
%   \num{0.4} & \myauthor & 01-06-2017 & PSK interoperability w/ WOLFSSL \\
%   \num{0.3} & \myauthor & 21-04-2017 & Integrated support for X.509 Certificates in tinyDTLS \\
%   \num{0.2} & \myauthor & 07-04-2017 & ECDHE ECDSA CHACHA20 POLY1305 SHA256 w/ tinyDTLS\\
%   \num{0.1} & \myauthor & 26-03-2017 & PSK CHACHA20 POLY1305 SHA256 w/ tinyDTLS \\
%   \hline
% \end{tabular}
\chapter*{}
{  \Huge  \textbf{Signatures} }\newline

\begin{center}
\textbf{P\textsc{rof}. D\textsc{r.} I\textsc{ng.} A\textsc{xel} S\textsc{ikora} (Scientific Director)}\\
\begin{figure}[H]
\centering
\includegraphics[height=7cm]{figures/titlepage_fig/signature.jpg}
\label{fig:Signature1}
\end{figure}
\textbf{D\textsc{r.} C\textsc{hadlia} J\textsc{erad}}
\begin{figure}[H]
\centering
\includegraphics[height=7cm]{figures/titlepage_fig/signature.jpg}
\label{fig:Signature1}
\end{figure}
\\

\end{center}
\newpage

\chapter*{Acknowledgments}
\label{cha:authors}
%\addxcontentsline{toc}{chapter}{Acknowledgments}

I have the pleasure to express my sincere gratitude to all the people who helped me and who made the successful
completion of this project possible.

I would like to offer my special thanks to Prof. Dr.-Ing. Axel Sikora for his trust and confidence and for giving me the opportunity
to join the Institut für verlässliche Embedded Systems und Kommunikationselektronik (ivESK).

I am highly indebted to my supervisor Dipl.-Phys. Andreas Walz for his guidance, encouragement and his highly valuable and
constructive suggestions during the realization of this work. I extremely appreciate the interesting discussions we
have had and I am highly thankful for his willingness to give his time so generously.

I would like to express my deep gratitude for my pedagogic supervisor Dr.-Ing. Chadlia Jerad for her supervision, assistance, support
and help as well as her important remarks and advice that helped to complete this project.

I would like also to thank the jury members for their time and efforts while reviewing my work. Besides, I am thankful to every
Professor who helped me in my academic career.

This work would not be possible without the support of my family, I would like to especially thank my mother Faouzia, my sister Baraa
and my brother Souhaib for their help and encouragement.

Last but not least, my deepest gratitude goes to Boutheina for her encouragement and support during this project.

\newpage

\begin{center}
\textbf{\Large{Abstract}}
\end{center}
In today’s automotive and industrial networks, data transmission is, for the most part, performed
without any special security measures. This security policy weakness can create unforeseen security
threats to the network, the network resources, and the data.
SafetyNETp is an Industrial Ethernet protocol used in system automation technology which does not provide
any security measures for protecting the network. The lack of security measures makes SafeyNETp vulnerable to several attacks.
The work presented in this project consists of designing and implementing an end-to-end DTLS based security layer for SafetyNETp.


%  The work presented in this project is
% twofold. In the first part, we integrate the \ac{DTLS} protocol over a SafetyNETp
% Network. In the second part, we evaluate that combination and test the software in a realistic
% environment. In order to accomplish these two parts, we need to study SafetyNETp and it's
% implementation, then we study DTLS protocol and find and analyze
% the existing implementations. Thereafter, we choose the most suitable implementation for our
% project to finally integrate DTLS with SafetyNETp.
%
% The work presented in this project consists of designing and implementing an end-to-end based security l
% \newline \textbf{Keywords :} Industrial Networks, SafetyNETp, \ac{DTLS}, C, Shell, Linux, Raspberry pi, ivESK, embedded system and software.

\begin{center}
\textbf{\Large{Résumé}}
\end{center}
Aujourd’hui, au sein des réseaux automobiles et industriels, la transmission de données est le plus souvent réalisée
sans mesures de sécurité spéciales. Cette faiblesse peut créer une menace de sécurité imprévue pour le réseau, les
ressources du réseau et les données. SafeyNETp est un protocole basé sur l'Ethernet industriel utilisé dans
l'industrie qui ne fournit aucune mesure de securité ce qui le rend
vulnérable à plusieurs attaques. Le travail présenté dans ce projet consiste à concevoir et implémenter une
couche de sécurité basé sur le protocole DTLS qui permet d'assurer une sécurité de bout en bout.

% Le travail présenté dans ce rapport
% consiste donc, dans sa première partie, à intégrer le protocole de sécurité pour le transport de
% données (DTLS) à un réseau SafeyNETp et ensuite, dans la deuxième partie, nous évaluons cette combinaison
% et nous testons le logiciel dans un environnement réaliste. Afin d'accomplir ces deux parties, nous devons
% étudier SafetyNETp et son implémentation, étudier le potocole DTLS et ensuite trouver et analyser les
% implémentations existantes. Par la suite, nous choisissons l'implémentation la plus appropriée à notre projet et enfin intégrer
% DTLS avec SafetyNETp.
% \newline \textbf{Mots clés :} Réseaux industriels, SafetyNETp, \ac{DTLS}, C, Shell, Linux, Raspberry pi, ivESK, systéme et logiciel embarqué.

\setcode{utf8}
\begin{arabtext}
\begin{center}
\textbf{\Large{مُلَخَّصٌ}}
\end{center}
في شبكات الإتصالات المستعملة في السيارات والصناعات اليوم، يتم  نقل البيانات في معظم الحالات دون أي تدابير أمنية خاصة. ضعف هذه السياسة الأمنية يمكن أن يخلق تهديدات على أمن الشبكة و الموارد و البيانات.
العمل المقدم في هذا المشروع يكمن في جزئين، في الجزء الاول نقوم بدمج بروتوكول حماية طبقة النقل \LR{DTLS} عبر شبكة \LR{SafetyNETp} ، في الجزء الثاني ، نقوم بتقييم ذلك الدمج واختبار البرنامج في بيئة واقعية.
من أجل تحقيق هذين الجزئين ، نحن بحاجة لدراسة \LR{SafetyNETp} وتطبيقها ، ودراسة \LR{DTLS} بروتوكول  ومن ثم العثور على التطبيقات الحالية المتوفرة ل \LR{DTLS} و من خلال ذلك نقوم  بإختياري  التطبيق الأنسب لنظامنا
و نقوم أخيراً بالدمج.
% الكلمات المفاتيح :
\end{arabtext}
